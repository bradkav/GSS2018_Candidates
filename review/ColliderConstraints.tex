\subsection{Constraints on Dark Matter from collider experiments}
As WIMPs interact very weakly with normal matter, high energy particle colliders like the LHC are unsuitable for directly detecting them. These experiments however can be useful in putting constraints on WIMP properties like its mass and cross-section, which can aid in direct and indirect searches for these particles. As direct detection experiments depend on a strong coupling of a WIMP to nucleons, this means that a significant amount of WIMPs will also be produced in high-energy collisions. In particle collisions transverse momentum $p_{T}$, so momentum perpendicular to the beam line, needs to be conserved. Missing transverse momentum could then indicate unidentified particles, including WIMPs.

Next we will discuss models that can be used to determine these constraints. We would like to know how in depth these models must be treated and if these sources are fine or if more up-to-date sources would be better.
https://arxiv.org/pdf/1005.1286.pdf
https://arxiv.org/abs/1105.3248
https://arxiv.org/pdf/hep-ph/0404175.pdf